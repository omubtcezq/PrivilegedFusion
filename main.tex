
%% bare_conf.tex
%% V1.4b
%% 2015/08/26
%% by Michael Shell
%% See:
%% http://www.michaelshell.org/
%% for current contact information.
%%
%% This is a skeleton file demonstrating the use of IEEEtran.cls
%% (requires IEEEtran.cls version 1.8b or later) with an IEEE
%% conference paper.
%%
%% Support sites:
%% http://www.michaelshell.org/tex/ieeetran/
%% http://www.ctan.org/pkg/ieeetran
%% and
%% http://www.ieee.org/

\documentclass[conference]{IEEEtran}

% Imports
\usepackage{local_macros/isasmathmacros}

% correct bad hyphenation here
\hyphenation{op-tical net-works semi-conduc-tor}


\begin{document}

% paper title
\title{Privileged Estimate Fusion With Correlated Gaussian Keystreams}

\author{\IEEEauthorblockN{Marko Ristic}
\IEEEauthorblockA{Autonomous Multisensor Systems Group (AMS),\\
Institute for Intelligent Cooperating Systems (ICS),\\
Otto von Guericke University (OVGU),\\
Magdeburg, Germany\\
Email: marko.ristic@ovgu.de}
\and
\IEEEauthorblockN{Benjamin Noack}
\IEEEauthorblockA{Autonomous Multisensor Systems Group (AMS),\\
Institute for Intelligent Cooperating Systems (ICS),\\
Otto von Guericke University (OVGU),\\
Magdeburg, Germany\\
Email: benjamin.noack@ovgu.de}}

% make the title area
\maketitle

% 
%        d8888 888888b.    .d8888b.  
%       d88888 888  "88b  d88P  Y88b 
%      d88P888 888  .88P  Y88b.      
%     d88P 888 8888888K.   "Y888b.   
%    d88P  888 888  "Y88b     "Y88b. 
%   d88P   888 888    888       "888 
%  d8888888888 888   d88P Y88b  d88P 
% d88P     888 8888888P"   "Y8888P"  
%                                    
%                                    
%                                    
% 

% As a general rule, do not put math, special symbols or citations in the abstract
\begin{abstract}
The abstract goes here.
\end{abstract}

% no keywords

\IEEEpeerreviewmaketitle


% 
% 8888888 888b    888 88888888888 8888888b.   .d88888b.  
%   888   8888b   888     888     888   Y88b d88P" "Y88b 
%   888   88888b  888     888     888    888 888     888 
%   888   888Y88b 888     888     888   d88P 888     888 
%   888   888 Y88b888     888     8888888P"  888     888 
%   888   888  Y88888     888     888 T88b   888     888 
%   888   888   Y8888     888     888  T88b  Y88b. .d88P 
% 8888888 888    Y888     888     888   T88b  "Y88888P"  
%                                                        
%                                                        
%                                                        
% 

\section{Introduction}
\begin{itemize}
  \item $3/4$ of a page including abstract. Can be relatively similar to the previous privilege paper.
  \item Role of estimation and increase in relevance of privacy and state secrecy.
  \item Usual methods hide all information, sometimes we want some leakage that can be used for a specific task.
  \item e.g. leakage of control inputs or leakage of information vector sums.
  \item Idea of privilege, e.g. GPS \cite{grovesPrinciplesGNSSInertial2015}, chaotic systems.
  \item Interested in cryptographic quantisation, provided by the previous paper which considers linear systems and uses the optimality of the Kalman filter. Doesn't consider the effect of multiple privilege-providing sensors and the effect of fusing of their measurements without keys to obtain better estimates. 
  \item Contribution stated explicitly.
  \item Use case of the scenario, perhaps something where measurements are infrequent so synchronisation isn't a problem, like weather sensors. Alternatively, something relating to privileged access to sensor features.
\end{itemize}

\subsection{Notation}
\begin{itemize}
  \item Matrices, vectors.
  \item Positive definitness.
  \item Pseudorandom distribution.
  \item Estimator in cryptographic sense.
  \item Negligible fuction and negligible covariance.
\end{itemize}

% 
% 8888888b.  8888888b.   .d88888b.  888888b.   
% 888   Y88b 888   Y88b d88P" "Y88b 888  "88b  
% 888    888 888    888 888     888 888  .88P  
% 888   d88P 888   d88P 888     888 8888888K.  
% 8888888P"  8888888P"  888     888 888  "Y88b 
% 888        888 T88b   888     888 888    888 
% 888        888  T88b  Y88b. .d88P 888   d88P 
% 888        888   T88b  "Y88888P"  8888888P"  
%                                              
%                                              
%                                              
% 

\section{Problem Formulation}
\begin{itemize}
  \item We are considering an environment where multiple sensors are present and required for the greatest estimation accuracy of the system they are measuring.
  \item We want to provide multiple levels of privilege to estimators, such that estimators with a higher level of privilege can achieve better estimates given the same measurements.
  \item We consider linear system and measurement models, given by the usual equations, which will make proving relevant cryptographic privileges straightforward. Each sensor $i$ holds its own secret key $sk_i$, which can be made available to an estimator of a suitable privilege.
  \item While we are interested in a cryptographic difference in estimation between estimators who do and do not hold sensor keys, respectively, the involvement of multiple sensors means that access to additional sensors and the fusing of measurements from sensors whose key is not known also need to be considered.
  \item We want to provide a scheme for estimation privileges that guarantees two types of estimation differences.
  \item Firstly, we want a lower bound on the difference between estimation performance of an estimator that holds no sensor keys (an unprivileged estimator) but has access to all measurements, and an estimator holding a subset of sensor keys (a privileged estimator) using only measurements from sensors to which they hold a key. This construction means the bound remains a lower-bound when the unprivileged estimator has access to fewer sensors or when the privileged estimator has access to more and exhaustively captures the benefits of knowing sensor keys.
  \item Secondly, we want an upper bound on the additional estimation performance a privileged estimator can achieve by fusing measurements from sensors to which they do not hold a key. The motivation behind this guarantee is that fusing additional measurements from sensors whose keys are not known should not provide as much estimation benefit as acquiring another sensor key, thus preserving the order of possible estimation performance across privileges.
  \item The construction of the resulting scheme should be such that two free parameters can be chosen to control the values of these two bounds, respectively.
\end{itemize}

% 
% 8888888b.  8888888b.  8888888888 888      8888888 888b     d888 
% 888   Y88b 888   Y88b 888        888        888   8888b   d8888 
% 888    888 888    888 888        888        888   88888b.d88888 
% 888   d88P 888   d88P 8888888    888        888   888Y88888P888 
% 8888888P"  8888888P"  888        888        888   888 Y888P 888 
% 888        888 T88b   888        888        888   888  Y8P  888 
% 888        888  T88b  888        888        888   888   "   888 
% 888        888   T88b 8888888888 88888888 8888888 888       888 
%                                                                 
%                                                                 
%                                                                 
% 

\section{Preliminaries}
\subsection{Cryptographic Estimation Privilege}
\begin{itemize}
  \item The formal definition of a privileged estimation scheme was introduced in (previous paper) and captures a reduction in estimation performance that can always be achieved for an estimator not knowing the scheme key compared to one knowing it and unmodified measurements.
  \item We will use a slight generalisation of this definition to capture an arbitrary difference in estimation that can always be achieved between estimators knowing and not knowing the scheme key, respectively.
  \item Give the definition of a privileged estimation scheme (Setup and Noise) but with modified Noise to allow the case that neither estimator has access to true measurements.
  \item 
  \item A series of covariances such that the difference between the best possible estimation from a privileged estimator and an unprivileged one is bounded by the series for all $k$.
  \item This now allows for an unprivileged estimator to have a potentially better estimate (but still bounded by how much better). Also, the matrices D in the definition itself no longer need to be valid covariances (since they can be 'negative').
\end{itemize}
\subsection{Gaussian keystream}
\begin{itemize}
  \item A stream of pseudorandom Gaussian samples which relies on a key for its generation. The samples are indistinguishable from a truly random stream of Gaussian samples to someone without the key, while someone with the key can reproduce the stream exactly.
  \item Equations for turning a stream cipher into a multivariate Gaussian stream.
  \item Note the assumption made about floating-point numbers and why it is reasonable to use them as truly randomly generated reals in cryptography proofs.
\end{itemize}


\section{Privileged Fusion}
\begin{itemize}
  \item The idea is to use correlated additive pseudorandom Gaussian noise at each sensor, which can only be removed from measurements produced by a specific sensor by an estimator holding the key for that sensor.
  \item To capture the correlation between measurements, we can consider the estimation problem of $n$ sensors as the stacked equation (stacked eq with modified $H$ and correlation matrix $C$).
  \item Similarly to the pseudorandom Gaussian multivariate stream, we can generate noises for each sensor with correlation $C$ by following the same process but using different keys to generate the standard Gaussians in the generation equation.
  \item Give the equation for generating Gaussian noise in the stacked model, and how the measurement at each sensor at time $k$ is modified accordingly.
  \item Computing this with an arbitrary $C^{1/2}$ however, would require an estimator to hold all $n$ keys to replicate the added noise locally before it can be removed. That is, each Gaussian in the resulting sensors noises vector $p_i$ may depend on standard Gaussians $z_i$ generated by all the other keys.
  \item Instead, finding a $C^{1/2}$ such that each $p_i$ can be computed sequentially given only the keys $<i$ allows removing noises from some sensors depending on the keys that are held. It does however restrict the subsets of keys that can be used to remove noises to sequential keys $i$, and therefore also restricts the privileges that are available to estimators. In this case, there are $n$ possible privileges, each holding one more key than the last (and allowing better estimation).
  \item These are the privilege levels and associated keys that we consider in this work and the cryptographic analysis ahead. An alternative method allowing for different subsets of keys to be sufficient for generating the relevant correlated noises are left for future work.
  \item We can now write the measurement equations for the measurements available at a privileged estimator holding a key subset $j$ as ($j$ non-noised measurement and $n-j$ noised ones - where the covariance is computed given the first $j$ variables).
  \item This contrasts the measurements equations for the unprivileged estimator (holding no keys) given by (single block equation, all sensors - or as many as they have access to).
  \item Intuitively, the correlation between added pseudorandom noise stops an unprivileged estimator from gaining too much information from fusing measurements, while the uncorrelation between them stops the using of one key available at a privileged estimator from being used to gain too much information from remaining measurements for which they do not hold a key.
\end{itemize}


\section{Cryptographic Privilege}
\begin{itemize}
  \item To prove the cryptographic privilege provided by the presented multisensor scheme, we will rely on the optimality of the linear Kalman filter to produce series' of covariances that are the best achievable (smallest possible) for a given estimator, and take the difference between estimators in question to bound their difference and achieve cryptographic estimation privilege.
  \item Similarly to the previous paper, the bounding series can be used in a cryptographic sketch proof which shows that the existence of an estimator violating the bound would imply the existence of a better linear estimator than the Kalman filter, known not to exist. This then guarantees the bound by contrapositive.
  \item We consider two types of unprivileged estimation which we want to bound, namely estimators holding no keys and estimators holding only a subset of keys.
\end{itemize}

\subsection{Unprivileged Adversaries}
\begin{itemize}
  \item If we assume an unprivileged estimator can access all $n$ sensors, then their stacked estimation model can be described by the appropriate equation at each timestep $k$. 
  \item We can then write the combined Kalman predict and update equation as the equation with params from the previous one.
  \item Due to the KF preserving error covariance order, by setting $P_0=0$ we get a series of covariance such that no unprivileged estimator can estimate with error covariances less than or equal to the series (lower-bound).
  \item Similarly we can do the same for an estimator to which we want to lower-bound the estimation difference with the unprivileged estimator. 
  \item A privileged estimator of privilege $j$ (access to the first $j$ keys) and that can only access the first $j$ sensors, has the appropriate measurement equation. Similarly, setting the initial covariance to zero gives the lower bound series on the best possible estimation error achievable by the privileged estimator using only the measurements from the sensors to which it holds keys.
  \item Taking the difference of the two estimator bounding series' produces the difference series.
  \item In the context of cryptographic privileged estimation scheme, the Setup and Noise algorithms are given accordingly. The optimality of the KF can then be used to achieve the security notion.
  \item In the above, we assume the unprivileged estimator has access to all $n$ sensors, while the $j$ privilege estimator has access to only the first $j$ sensors. In the case when the unprivileged estimator has access to fewer sensors or the privileged one to more, their difference in estimation can only increase, thus keeping the computed lower-bound a lower-bound and the cryptographic guarantee does not change (albeit the definitions of Setup and Noise will to capture the now available sets of measurements).
\end{itemize}

\subsection{Privileged Adversaries}
\begin{itemize}
  \item We can use a similar approach to guarantee the largest possible benefit in estimation available to an estimator of privilege $j$ (access to first $j$ keys) by fusing measurements from sensors to which they do not hold keys to get a better state estimate.
  \item We can again write stacked estimation models for the two estimators and use the optimality of the KF to give their best possible performances as a series of covariances.
  \item Taking the difference of the bounds now gives a bound on how much better an estimate can become when unprivileged measurements are fused to the privileged ones.
  \item In the context of a cryptographic privileged estimation scheme, the Setup and Noise algorithms can be given as follows. KF optimality can then be used to achieve the security notion.
  \item We note that unlike in the unprivileged adversary case, or the previous paper, here we do not use unmodified measurements for the privileged estimates and the resulting algorithms provide better estimation for the adversary than the privileged estimator. In this form, we show bound how much better an adversary can estimate when using unprivileged estimates resulting in series $D_k$ consisting of negative-definite rather than positive definite matrices.
\end{itemize}


\section{Simulation}
\begin{itemize}
  \item In addition to showing how to derive the bounds to the benefits of using unprivileged measurements, we have simulated concrete scenarios to demonstrate the methods.
  \item We consider the following linear system, and linear measurement models for $5$ sensors. Correlated and uncorrelated noise covariances are given by $Y$ and $Z$.
  \item Implementation details.
  \item First figure shows $4$ plots. Each plot shows the traces of estimator covariances for the unprivileged estimator holding no keys, a privileged estimator holding $1,2,3$ and $4$ keys, respectively (estimating only using the privileged measurements) and the difference between the privilege and unprivileged traces. The difference here is equal to the trace of the difference series given in the previous section (for each of the $4$ privileges).
  \item The second figure will again show $4$ plots. Now, each plot will show the traces of estimate covariances for the privileged estimator holding $1,2,3$ and $4$ keys, respectively, estimating using only the privileged measurements and estimating with all measurements (fusing unprivileged ones as well). The difference between the two traces will be plotted as well, which will in the case be negative (as the difference here is negative definite) and again be equal to a difference series from the previous section.
\end{itemize}


\section{Conclusion}
\begin{itemize}
  \item Concluding remarks.
  \item Future work includes exploring key subsets that do not need to be sequential and decentralised methods for multi-key correlated noise generation.
\end{itemize}

% conference papers do not normally have an appendix

% trigger a \newpage just before the given reference
% number - used to balance the columns on the last page
% adjust value as needed - may need to be readjusted if
% the document is modified later
%\IEEEtriggeratref{8}
% The "triggered" command can be changed if desired:
%\IEEEtriggercmd{\enlargethispage{-5in}}

% 
% 8888888b.  8888888888 8888888888 .d8888b.  
% 888   Y88b 888        888       d88P  Y88b 
% 888    888 888        888       Y88b.      
% 888   d88P 8888888    8888888    "Y888b.   
% 8888888P"  888        888           "Y88b. 
% 888 T88b   888        888             "888 
% 888  T88b  888        888       Y88b  d88P 
% 888   T88b 8888888888 888        "Y8888P"  
%                                            
%                                            
%                                            
% 

\bibliographystyle{IEEEtran}
\bibliography{bibliography/PrivilegedFusion}

% that's all folks
\end{document}


